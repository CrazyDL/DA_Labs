\documentclass[12pt]{article}

\usepackage{fullpage}
\usepackage{multicol,multirow}
\usepackage{tabularx}
\usepackage{ulem}
\usepackage[utf8]{inputenc}
\usepackage[russian]{babel}
\textheight=24cm

\begin{document}

\section*{Лабораторная работа №\,4 по курсу дискретного анализа: Поиск образца в строке}

Выполнил студент группы 08-207 МАИ \textit{Лебедев Иван}.

\subsection*{Условие}

\begin{enumerate}
\item Необходимо реализовать один из стандартных алгоритмов поиска образцов для указанного алфавита.
\item Поиск одного образца-маски: в образце может встречаться «джокер», равный любому другому символу. При реализации следует разбить образец на несколько, не содержащих «джокеров», найти все вхождения при помощи алгоритма Ахо-Корасик и проверить их относительное месторасположение.
\end{enumerate}

\subsection*{Метод решения}
\par В первой строке считывается образец, делится по джокерам на подобразцы, и по ним строится trie. После добавления всех подобразцов в trie происходит его прошивка. Затем считывается весь текст и в нём производится поиск. При поиске создается вектор равный размеру текста и заполняется нулями. При нахождении любого образца в этом векторе элемент под индексом начала подобразца в тексе увеличивался на единицу. Весь образец считается найденным если вектор[i] равен количеству подобразцов.

\subsection*{Описание программы}

\begin{enumerate}
\item\textsl{Заголовочные файлы:}\\  ТNode.h(Объявление узла и функций)
\item\textsl{Cpp файлы:}\\ ТNode.cpp(Описание функций объявленных в TNode.h) \\ main.cpp(Считывание данных, управление программой)
\end{enumerate}

\subsection*{Дневник отладки}

Программа не проходила тест №3 из-за ошибки при подсчете индексов, и как следствие выход за границы массива. Так же была ошибка при нумерации строк(не считались пустые строки).

\subsection*{Выводы}
С помощью алгоритма Ахо-Корасика можно искать сразу несколько образцов в строке за линейную сложность. Благодаря этому алгоритм обрел большую популярность среди программистов.
Время работы также зависит от организации данных. Если таблицу переходов в дереве хранить как индексный массив — увеличивается расход памяти, но уменьшается вычислительная сложность. Если таблицу переходов хранить как красно-чёрное дерево — расход памяти снижается, однако вычислительная сложность увеличивается.

\end{document}

