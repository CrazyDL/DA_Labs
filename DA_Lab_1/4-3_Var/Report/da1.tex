\documentclass[12pt]{article}

\usepackage{fullpage}
\usepackage{multicol,multirow}
\usepackage{tabularx}
\usepackage{ulem}
\usepackage[utf8]{inputenc}
\usepackage[russian]{babel}
\textheight=23cm

\begin{document}

\section*{Лабораторная работа №\,1 по курсу дискретного анализа: сортировка за линейное время}

Выполнил студент группы 08-207 МАИ \textit{Лебедев Иван}.

\subsection*{Условие}

\begin{enumerate}
\item Требуется разработать программу, осуществляющую ввод пар «ключ-значение», их упорядочивание по возрастанию ключа указанным алгоритмом сортировки за линейное время и вывод отсортированной последовательности.
\item Поразрядная сортировка. Типы ключей: Даты в формате DD.MM.YYYY, типы значений:Числа от 0 до $2^{64} - 1$.
\end{enumerate}

\subsection*{Метод решения}

Считываются  пары «ключ значение» в структуру, в которой дата храниться в виде строки для вывода и так же преобразуется в число для сортировки. После считывания всех данных, массив сортируется с помощью поразрядной сортировки, в которой высчитывается размер разряда, и в качестве внутренней сортировки используется сортировка подсчетом. Затем отсортированный массив выводится на экран.

\subsection*{Описание программы}

\begin{enumerate}
\item\textsl{Cpp файлы:}\\ main.cpp(Считывание и вывод данных, управление программой, поразрядная сортировка)
\end{enumerate}

\subsection*{Дневник отладки}

Программа неверно сортировала данные, из-за ошибки в алгоритме, проблема была устранена правкой алгоритма. 


\subsection*{Выводы}

Поразрядная сортировка наиболее эффективна в тех случаях, когда входные данные приблизительно одинаковой длинны. И количество разрядов меньше чем количество сортируемых элементов. Так же важным моментов является то, что используемая внутренняя сортировка должна быть устойчивой.

\end{document}

