\documentclass[12pt]{article}

\usepackage{fullpage}
\usepackage{multicol,multirow}
\usepackage{tabularx}
\usepackage{ulem}
\usepackage[utf8]{inputenc}
\usepackage[russian]{babel}
\textheight=23cm

\begin{document}

\section*{Лабораторная работа №\,1 по курсу дискртного анализа: сортировка за линейное время}

Выполнил студент группы 08-208 МАИ \textit{Левштанов Денис}.

\subsection*{Условие}

\begin{enumerate}
\item Требуется разработать программу, осуществляющую ввод пар «ключ-значение», их упорядочивание по возрастанию ключа указанным алгоритмом сортировки за линейное время и вывод отсортированной последовательности.
\item Карманная сортировка. Тип ключей: вещественные числа в промежутке [—100, 100], тип значений: строки переменной длины (до 2048 символов).
\end{enumerate}

\subsection*{Метод решения}

Считываются  пары «ключ значение» в структуру, эта структура помещается в вектор(собственная реализация). После считывания всех данных, вектор сортируется: создается вектор, состоящий из других векторов, размером равным количеству введённых элементов. Каждый элемент добавляется в вектор под индексом [ключ + 100(длинна свига с -100;100 на 0;200)/201(количество элементов на отрезке -100;100)  * количество введённых элементов], тем самым смещая данный отрезок -100;100 на отрезок 0;1, который требуется для карманной сортировки. Затем каждый "карман" вектора сортируется вставкой, после всё соединяется в один вектор и выводится на экран.

\subsection*{Описание программы}

\begin{enumerate}
\item\textsl{Заголовочные файлы:}\\ ЕVector.h(описание вектора) \\Element.h(структура с полями double Key, char* Value)\\ BucketSort.h(Объявление карманной сортировки).\\
\item\textsl{Cpp файлы:}\\ main.cpp(Считывание и вывод данных, управление программой)\\ BucketSort.cpp (Реализация карманной сортировки, сортировка вставкой)\\ TVector.cpp(Реализация вектора).
\end{enumerate}

\subsection*{Дневник отладки}

Программа не проходила тест по памяти из-за считывания строки переменной длинны в переменную с выделенной памятью в 2048 символов. Решение было найдено такое, что строка считывается во временную переменную с выделенной памятью в 2048 символов, затем считается длинна этой строки, в структуре выделяется для неё память и строка копируется в структуру. 

\subsection*{Тест производительности}
\begin{enumerate}
\item 100 000 элементов: \\ Time of working bucket: 0.062256\\
Time of working std: 0.077026
\item 1 000 000 элементов: \\ Time of working bucket: 0.75178 \\
Time of working std: 0.966112
\item 5 000 000 элементов: \\ Time of working bucket: 5.59361 \\
Time of working std: 7.78683
\end{enumerate}

\subsection*{Выводы}

Карманная сортировка применяется только в тех случаях когда известен сортируемый отрезок и элементы равномерно распределены на этом отрезке.

\end{document}

