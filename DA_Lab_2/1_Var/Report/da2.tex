\documentclass[12pt]{article}

\usepackage{fullpage}
\usepackage{multicol,multirow}
\usepackage{tabularx}
\usepackage{ulem}
\usepackage[utf8]{inputenc}
\usepackage[russian]{babel}
\textheight=25cm

\begin{document}

\section*{Лабораторная работа №\,2 по курсу дискретного анализа: Словарь}

Выполнил студент группы 08-207 МАИ \textit{Лебедев Иван}.

\subsection*{Условие}

\begin{enumerate}
\item Необходимо создать программную библиотеку, реализующую указанную структуру данных, на основе которой разработать программу-словарь. В словаре каждому ключу, представляющему из себя регистро-независимую последовательность букв английского алфавита длиной не более 256 символов, поставлен в соответствие некоторый номер, от 0 до 264 - 1. Разным словам может быть поставлен в соответствие один и тот же номер.
\item AVL-дерево.
\end{enumerate}

\subsection*{Метод решения}
В цикле считываются данные, проверяется первый символ, если он равен «+», то считывается пара «ключ значение» и вставляется в дерево. Если  такой ключ уже существует выводится «Exist».  При «-» считывается ключ и удаляется из дерева, если такого ключа в дереве нет, выводится «NoSuchWord». При «!» считывается следующее слово, если это «Save», то считывается имя файла и происходит сериализация существующего дерева. Если «Load» то считывается имя файла, создается новое дерево и происходит десериализация данных из файла. При неудаче работа продолжается с исходным деревом. Если первый символ введенного слова не совпадает ни с одним из вышеперечисленных, то выполняется поиск этого слова в дереве. 

\subsection*{Описание программы}

\begin{enumerate}
\item\textsl{Заголовочные файлы:}\\ TNode.h(Объявление узла AVL-дерева)\\ TAVLtree.h(Объявление функций AVL-дерева). 
\item\textsl{Cpp файлы:}\\ main.cpp(Считывание и вывод данных, управление программой)\\ TAVLtree.cpp (Реализация функций дерева)\\ TNode.cpp(Реализация функций узла).
\end{enumerate}

\subsection*{Дневник отладки}

Программа не проходила тест из-за того, что в некоторых случаях строка не переводилась в нижний регистр.

\subsection*{Выводы}

АВЛ-дерево — сбалансированное по высоте двоичное дерево поиска. В отличие от обычного бинарного дерева поиска, в АВЛ-дереве все операции в наихудшем случае выполняются за logn. в тоже время на достаточно простое для понимания и реализации.

\end{document}

