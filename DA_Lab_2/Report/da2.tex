\documentclass[12pt]{article}

\usepackage{fullpage}
\usepackage{multicol,multirow}
\usepackage{tabularx}
\usepackage{ulem}
\usepackage[utf8]{inputenc}
\usepackage[russian]{babel}
\textheight=23cm

\begin{document}

\section*{Лабораторная работа №\,2 по курсу дискретного анализа: Словарь}

Выполнил студент группы 08-208 МАИ \textit{Левштанов Денис}.

\subsection*{Условие}

\begin{enumerate}
\item Необходимо создать программную библиотеку, реализующую указанную структуру данных, на основе которой разработать программу-словарь. В словаре каждому ключу, представляющему из себя регистронезависимую последовательность букв английского алфавита длиной не более 256 символов, поставлен в соответствие некоторый номер, от 0 до 264 - 1. Разным словам может быть поставлен в соответствие один и тот же номер.
\item Б-дерево.
\end{enumerate}

\subsection*{Метод решения}
В цикле считывать слова, подающиеся на вход, проверить первый символ, если он равен «+», то считать пара «ключ значение» в класс TElem, и этот элемент вставить в Б-дерево. Если  такой ключ уже существует вывести «Exist».  При «-» считать ключ и удалить его в дереве, если такого ключа в дереве нет, вывести «NoSuchWord». При «!» считать следующее слово, если это «Save», то считать имя файла и сериализировать существующее дерево. Если «Load» то считать имя файла, создать новое дерево и десериализировать данные из файла. При неудаче работа продолжается с исходным деревом. Если перый символ введенного слова не совпадает ни с одним из вышеперечисленных, то выполнить поиск этого слова в дереве. Также на протежении всей программы обрабатываются и выводятся такие ошибки как: memory error, file error, deserialization error.

\subsection*{Описание программы}

\begin{enumerate}
\item\textsl{Заголовочные файлы:}\\ TVector.h(описание и реализация собвственного вектора)
 \\TSmartPointer.h(описание и реализация собвственных умных указателей)\\ TNode.h(Объявление узла Б-дерева)\\ TBTree.h(Объявление Б-дерева). 
\item\textsl{Cpp файлы:}\\ main.cpp(Считывание и вывод данных, управление программой)\\ TBTree.cpp (Реализация вставки, удаления, поиска елемента, сериализации, десериализации дерева)\\ TNode.cpp(Реализация конструкторов узла).
\end{enumerate}

\subsection*{Дневник отладки}

Программа не проходила тест №4 из-за ошибки в алгоритме удаления. На тесте №8 программа выдавала неверный ответ из за считывания данных не в адрес строки, а в адрес указателя. На 11 теста был получен runtime error signal 6 по нехватке памяти, из-за считывания строки переменной длинны в переменную с выделенной памятью в 257 символов. Проблема была решена динамическим выделением памяти под строку. На 14 тесте программа не проходила по времени, ускорить программу получилось переписав поиск элемента в узле с обычного линейного на бинарный. 

\subsection*{Тест производительности}
\begin{enumerate}
	\item 200 000 строк без сериализации: \\ Time of working: 1.393
	\item 400 000 строк без сериализации: \\ Time of working: 2.787
	\item 50 000 строк с сериализацией: \\ Time of working: 6.279
	\item 100 000 строк с сериализацией: \\ Time of working: 25.568
	
\end{enumerate}
\subsection*{Выводы}

Б-дерево - это сбалансированное, сильно ветвистое дерево. Дерево я реализовывал таким образом, что оно целиком хранится в оперативной памяти, но обычно Б-дерево используется во внешней памяти, то есть каждый узел записывается на диск,а по мере необходимости нужные узлы считываются с диска, и в итоге в оперативной памяти находится только рабочий узел, что помогает работать с большими объемами данных, не забивая оперативную память.

\end{document}

