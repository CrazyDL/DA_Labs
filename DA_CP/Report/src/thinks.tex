\section{Выводы}

\par Выполнив курсовую работу, я понял то, как устроен звук в цифровом понимании, из чего он состоит, на что влияет каждый из параметров. Также я изучил 2 библиотеки по работе с аудио. Первая была ffmpeg -- это целый набор свободных библиотек, которые позволяют записывать, конвертировать и передавать цифровые аудио- и видеозаписи в различных форматах. Но она оказалась слишком громоздкой и ее функционал был слишком избыточен для моей задачи. Поэтому я использовал mpg123 -- это библиотека, которая имеет меньший функционал чем ffmpeg, но она полностью удовлетворяет все требования моей задачи.

Так же я понял, как примерно работает популярное приложение shazam, которое помогает распознавать музыку. Принцип его работы не так сложен, как это кажется на первый взгляд, но в тоже время существует множество нюансов, которые следует учитывать.

Вообще музыка играет огромную роль в жизни человека, и с помощью алгоритмов можно сделать не только жизнь людей чуточку проще, распознавая музыку, которую они где-то услышали, но также можно исследовать произведения на плагиат, искать исполнителей, которые вдохновляли некоторых первопроходцев в каком-либо жанре музыки и тд.


\pagebreak