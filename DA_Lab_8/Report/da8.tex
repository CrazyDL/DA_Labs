\documentclass[12pt]{article}

\usepackage{fullpage}
\usepackage{multicol,multirow}
\usepackage{tabularx}
\usepackage{ulem}
\usepackage[utf8]{inputenc}
\usepackage[russian]{babel}
\textheight=23cm

\begin{document}

\section*{Лабораторная работа №\,8 по курсу дискретного анализа: Жадные алгоритмы}

Выполнил студент группы 08-208 МАИ \textit{Левштанов Денис}.

\subsection*{Условие}

\begin{enumerate}
\item Разрабтать жадный алгоритм решения задачи, определяемой своим вариантом. Доказать его корректность, оценить скорость и объём затрачиваемой оперативной памяти.

Реализовать программу на языке C или C++, соответствующую построенному алгоритму.
\item Максимальный треугольник

Заданы длины N отрезков, необходимо выбрать три таких отрезка, которые образовывали бы треугольник с максимальной площадью. Формат входных данных: на первой строке находится число N, за которым следует N строк с целыми числами-длинами отрезков. Формат выходных данных: если никакого треугольника из заданных отрезков составить нельзя — 0, в противном случае на первой строке площадь треугольника с тремя знаками после запятой, на второй строке — длины трёх отрезков, составляющих этот треугольник. Длины должны быть отсортированы.

\end{enumerate}

\subsection*{Метод решения}
\par Считывается количество элементов, и создается вектор для них. Вектор сортируется в порядке убывания. Затем начиная с самых больших элементов берутся три элемента и проверяются на возможность составить из них треугольник (каждая сторона должна быть меньше чем сумма других двух), если это условие выполняется то высчитывается площадь по формуле Герона и выводится ответ. Общее время работы О(nlogn + n) nlogn для сортировки и n для прохода по отсортированной последовательности.

\subsection*{Описание программы}

\begin{enumerate}
\item main.cpp(Считывание данных, управление программой):
\par size\_t n - количество элементов
\par vector<int64\_t> vec - вектор элементов
\par double square - площадь
\end{enumerate}

\subsection*{Дневник отладки}

Программа не проходила 5 тест из-за того, что не обрабатывалась последняя тройка элементов в массиве.

\subsection*{Тест производительности}

\begin{enumerate}
	\item Time of working 1000: 0.002
	\item Time of working 10000: 0.019
	\item Time of working 100000: 0.168
	\item Time of working 1000000: 1.774
	
\end{enumerate}
\subsection*{Выводы}
\par Решение с помощью жадных алгоритмов заключается в том, что на каждом этапе решения выбирается оптимальное, в отличие от динамического программирования, где осуществляется просмотр всех возможных вариантов решений. Тем самым можно значительно улучшить асимптотику многих задач. Кроме того, практически все задачи, решаемые жадными алгоритмами можно решить с помощью динамического программирования, но не наоборот.

\end{document}

