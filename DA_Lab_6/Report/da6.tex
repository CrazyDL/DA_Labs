\documentclass[12pt]{article}

\usepackage{fullpage}
\usepackage{multicol,multirow}
\usepackage{tabularx}
\usepackage{ulem}
\usepackage[utf8]{inputenc}
\usepackage[russian]{babel}
\textheight=23cm

\begin{document}

\section*{Лабораторная работа №\,6 по курсу дискретного анализа: Калькулятор}

Выполнил студент группы 08-208 МАИ \textit{Левштанов Денис}.

\subsection*{Условие}

\begin{enumerate}
\item Необходимо разработать программную библиотеку на языке C или C++, реализующую простейшие арифметические действия и проверку условий над целыми неотрицательными числами. На основании этой библиотеки, нужно составить программу, выполняющую вычисления над парами десятичных чисел и выводящую результат на стандартный файл вывода.

\end{enumerate}

\subsection*{Метод решения}
\par Считываются два числа и оператор, который к ним следует применить. Затем определяется что это за оператор и с помощью перегрузки операторов вычисляется значение и выводится на экран. При этом, если какая-либо операция "выбросила" исключение, то оно ловится и на экран выводится строка "Error".

\subsection*{Описание программы}

\begin{enumerate}
\item TBigUInt.h(Объявление класса длинного без знакового числа):
\par vector<int64\_t> number; - вектор где хранится число
\par size\_t size; - размер числа
\par static const int64\_t BASE = 1000000000; - основание по которому разбивается число
\par int static const size\_t NUM\_SIZE = 9; - количество цифр в основание
\item TBigUInt.cpp(Определение класса объявленного в TBigUInt.h)
\par int64\_t Get(size\_t i) const; - возвращает i элемент вектора если i меньше его размера, иначе возвращает 0
\par void FixZero(); - удаляет ведущие нули и считает размер числа
\par void Clear(); - очищает число
\item main.cpp(Считывание данных, управление программой)
\end{enumerate}

\subsection*{Дневник отладки}

Программа не проходила тесты 2, 6, 10 из-за ошибок в алгоритме.

\subsection*{Тест производительности}

\begin{enumerate}
	\item Time of working 100 operations: 7.236
	\item Time of working 500 operations: 59.639
	\item Time of working 1000 operations: 92.537
	
\end{enumerate}
\subsection*{Выводы}
Большинство языков программирования имеют типы данных ограниченные по размеру до $2^{64}$, но они не способны оперировать с числами в которых больше 20 цифр. Решением этой проблемы является использование длинной арифметики. В некоторых языках программирования она встроена по умолчанию(python), а в c++ приходится либо использовать готовые библиотеки, либо реализовывать её самому. Очевидно что операции с длинными числами значительно медленнее чем с обычными и памяти они используют больше, так операции реализуются не аппаратно, а программно.

\end{document}

