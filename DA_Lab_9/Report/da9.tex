\documentclass[12pt]{article}

\usepackage{fullpage}
\usepackage{multicol,multirow}
\usepackage{tabularx}
\usepackage{ulem}
\usepackage[utf8]{inputenc}
\usepackage[russian]{babel}
\textheight=23cm

\begin{document}

\section*{Лабораторная работа №\,9 по курсу дискретного анализа: Графы}

Выполнил студент группы 08-208 МАИ \textit{Левштанов Денис}.

\subsection*{Условие}

\begin{enumerate}
\item Разработать программу на языке C или C++, реализующую указанный алгоритм. Формат входных и выходных данных описан в варианте задания. Первый тест в проверяющей системе совпадает с примером.
\item Поиск кратчайших путей между всеми парами вершин алгоритмом Джонсона

Задан взвешенный ориентированный граф, состоящий из n вершин и m ребер. Вершины пронумерованы целыми числами от 1 до n. Необходимо найти длины кратчайших путей между всеми парами вершин при помощи алгоритма Джонсона. Длина пути равна сумме весов ребер на этом пути. Обратите внимание, что в данном варианте веса ребер могут быть отрицательными, поскольку алгоритм умеет с ними работать. Граф не содержит петель и кратных ребер.

\end{enumerate}

\subsection*{Метод решения}
\par Считывается количество вершин и ребер, затем считывается существующие пути и их вес. В алгоритме Джонсона строится новый граф с одной дополнительной вершиной и для нее запускается алгоритм Беллмана-Форда. При наличии отрицательного цикла программа завершает работу, иначе с помощью вычисленных значений веса в исходном графе делаются неотрицательными и для каждой вершины запускается алгоритм Дейкстры. В конце вес путей нормализуется.

\subsection*{Описание программы}

\begin{enumerate}
\item struct TEdge(структура для хранения ребра)
\item struct TGraph(структура для хранения графа)
\item void Dijkstra(AdjMatrix\& gr, size\_t givenNode, AdjMatrix\& dist, size\_t n)(алгоритм Дейкстры)
\item bool BellmanFord(TGraph\& gr, size\_t givenNode, AdjMatrix\& dist)(алгоритм Беллмана-Форда)
\item bool Johnson(TGraph\& gr, AdjMatrix\& dist)(алгоритм Джонсона)
\item main.cpp(Считывание данных, управление программой):

\end{enumerate}

\subsection*{Дневник отладки}

Программа не проходила 16 тест из-за нехватки времени и 19 теста из-за нехватки памяти. Решилось все использованием другой структуры для хранения графа и оптимизации программы.


\subsection*{Выводы}
\par Поиск кратчайших путей в графе очень распространенная и используемая в реальной жизни задача и существуют множество алгоритмов для ее решения. Например, есть два алгоритма (Дейкстры и Беллмана-Форда) для поиска путей из одной вершины в остальные. Однако у каждого из них свои недостатки - Дейкстра не работает с отрицательными весами, а Беллмана-Форд медленнее первого. Алгоритм Джонсона объединяет эти 2 алгоритма делая общую сложность $O(V^2logE + VE)$

\end{document}

